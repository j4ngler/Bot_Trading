% -*- coding: utf-8 -*-
\documentclass[12pt,a4paper]{article}
\usepackage[utf8]{inputenc}
\usepackage[vietnamese]{babel}
\usepackage{amsmath,amssymb}
\usepackage{geometry}
\usepackage{fancyhdr}
\usepackage{booktabs}
\usepackage{graphicx}
\usepackage{xcolor}
\usepackage{enumitem}

\geometry{margin=2.3cm}
\pagestyle{fancy}
\fancyhf{}
\fancyhead[L]{Buổi 3 - AI Advisor \& Báo cáo}
\fancyhead[R]{Trading Bot - Testnet}
\fancyfoot[C]{\thepage}

\title{\textbf{Giáo Án Buổi 3}\\
\large ChatGPT Advisor, Auto-Reporting \& Đánh Giá Hiệu Suất}
\author{}
\date{}

\begin{document}

\maketitle

\section*{1. Mục tiêu buổi học}
\begin{itemize}[leftmargin=1.2cm]
    \item Hiểu cách \texttt{ChatGPTAdvisor} tạo khuyến nghị và giới hạn của AI trong giao dịch.
    \item Thực hành chỉnh prompt, đọc log phản hồi, đo ``độ tin cậy'' từ mô hình.
    \item Kết nối chuỗi: AI Advisor $\rightarrow$ Risk Manager $\rightarrow$ báo cáo \texttt{reporting\_monitoring.py}.
    \item Hoàn thiện dashboard (tab Báo Cáo + Chat) và xuất báo cáo \texttt{duong\_cong\_von.png}, \texttt{trading\_report.html}.
\end{itemize}

\section*{2. Chuẩn bị}
\begin{itemize}[leftmargin=1.2cm]
    \item Máy đã cài \texttt{requirements.txt}, có tài khoản OpenAI (hoặc key demo từ giáo viên).
    \item File \texttt{.env} mẫu: \texttt{OPENAI\_API\_KEY}, \texttt{BINANCE\_API\_KEY}, \texttt{BINANCE\_SECRET\_KEY}.
    \item Slide tóm tắt kiến trúc AI Advisor + Reporting (sơ đồ module trong README).
    \item Dữ liệu mẫu trong \texttt{data/trading\_history.db} để HS tạo báo cáo ngay cả khi chưa giao dịch.
\end{itemize}

\section*{3. Lộ trình 90 phút}

\subsection*{A. Warm-up: ``AI có sai không?'' (5')}
\textbf{Hoạt động:} Chiếu 3 log ChatGPT (BUY, SELL, HOLD) và hỏi: ``Nếu AI khuyến nghị BUY liên tục thì Risk Manager xử lý sao?'' \\
\textbf{Trả lời gợi ý:} Risk Manager chỉ cho phép lệnh khi các điều kiện vẫn đạt: RSI không quá 70, ATR/giá không vượt ngưỡng biến động, độ tin cậy (confidence) $\ge$ ngưỡng yêu cầu và chưa vượt giới hạn vị thế. Nếu một trong các điều kiện thất bại, lệnh vẫn bị chặn dù AI liên tục BUY. \\
\textbf{Mục tiêu} giúp HS nhắc lại rằng AI chỉ là bộ phân tích, cuối cùng vẫn cần luật quản trị rủi ro và báo cáo để kiểm tra.

\subsection*{B. ChatGPT Advisor deep-dive (25')}
\textbf{Bước 1: Giải phẫu file \texttt{chatgpt\_advisor.py} (8')} \\
GV hướng dẫn đọc:
\begin{enumerate}[leftmargin=0.8cm]
    \item Hàm \texttt{\_\_init\_\_}: tạo OpenAI client, lấy model từ \texttt{config.OPENAI\_MODEL}.
    \item Hàm \texttt{analyze\_market}: dựng prompt từ \texttt{TRADING\_PROMPT}, gửi API, parse BUY/SELL/HOLD.
    \item Hàm \texttt{chat\_with\_user}: duy trì \texttt{history}, giới hạn 40 message.
\end{enumerate}

\textbf{Bước 2: Chỉnh prompt và thử nghiệm (10')} \\
\begin{itemize}[leftmargin=0.6cm]
    \item HS mở \texttt{config.py}, thay đổi câu ``Luôn nhắc nhở về rủi ro'' thành thông điệp của nhóm.
    \item Chạy \texttt{python run.py} \rightarrow Tab Chat, gửi câu hỏi ``BTC đang xu hướng gì?'' để xem phản hồi mới.
    \item Quan sát log trong panel Logs: thảo luận khi nào nên tin khuyến nghị AI.
\end{itemize}

\textbf{Bước 3: Đọc ``confidence'' (7')} \\
\begin{itemize}[leftmargin=0.6cm]
    \item Trong \texttt{run\_once()}, in thêm dòng \texttt{advice['confidence']} để so sánh với giá trị Risk Manager yêu cầu.
    \item Bài tập: thay điều kiện confidence $\ge 60$ trong \texttt{risk\_manager.py} và xem AI còn bao nhiêu lệnh được phép.
\end{itemize}

\subsection*{C. Tự động hóa báo cáo (25')}
\textbf{Bước 1: Tạo báo cáo từ DB (10')} \\
\begin{itemize}[leftmargin=0.6cm]
    \item Chạy script:
\begin{verbatim}
python -m src.reporting_monitoring
\end{verbatim}
    \item HS kiểm tra thư mục \texttt{data/} có \texttt{trading\_report.html} và \texttt{duong\_cong\_von.png}.
    \item Thảo luận: PnL âm/ dương thể hiện gì? Equity curve cho thấy xu hướng vốn ra sao?
\end{itemize}

\textbf{Bước 2: Tùy biến tab Báo cáo trong GUI (10')} \\
\begin{itemize}[leftmargin=0.6cm]
    \item Mở \texttt{gui\_app.py}, tìm hàm \texttt{refresh\_report()} và \texttt{update\_chart()}.
    \item HS sửa màu sắc/tiêu đề để nhấn mạnh ``📈 Đường Cong Vốn''.
    \item Bấm ``Làm mới báo cáo'' trong GUI để thấy ảnh mới.
\end{itemize}

\textbf{Bước 3: Bài tập mini (5')} \\
Mỗi nhóm thêm một dòng chú thích dưới biểu đồ: ``Cập nhật lúc HH:MM'' (gợi ý dùng \texttt{datetime.now()} trong \texttt{update\_chart}).

\subsection*{D. Case study: ``AI đưa lệnh sai'' (20')}
\begin{enumerate}[leftmargin=0.9cm]
    \item GV cung cấp log giả: ChatGPT BUY nhưng giá giảm mạnh.
    \item HS dùng báo cáo 30 ngày để tìm PnL lệnh đó, xác định vì sao Risk Manager vẫn cho phép (đủ điều kiện?).
    \item Nhóm đề xuất cách cải thiện: tăng yêu cầu confidence, thêm chỉ báo phụ, chỉnh stop-loss trong \texttt{risk\_manager}.
\end{enumerate}
Kết thúc bằng bảng ``Hành động khắc phục'':
\begin{itemize}[leftmargin=0.6cm]
    \item \textbf{Triệu chứng}: AI BUY sai.
    \item \textbf{Nguyên nhân}: RSI trong vùng nhiễu, ATR thấp nên Risk Manager OK.
    \item \textbf{Giải pháp}: thêm checklist kiểm tra xu hướng, xem lại prompt AI.
\end{itemize}

\subsection*{E. Tổng kết \& checklist Git (5')}
\begin{itemize}[leftmargin=0.6cm]
    \item HS ghi lại thay đổi đã làm: prompt AI, ngưỡng confidence, giao diện báo cáo.
    \item Commit tạm với thông điệp \texttt{"lesson-3: ai advisor \& reporting"}.
    \item Nhắc nhở: luôn ẩn API key khi share repo.
\end{itemize}

\section*{4. Bài tập về nhà}
\begin{itemize}[leftmargin=1.2cm]
    \item Viết đoạn script trong \texttt{scripts/} tự động gửi email báo cáo HTML mỗi 24h (gợi ý dùng \texttt{smtplib} hoặc chỉ cần lưu file).
    \item Thiết kế thêm một câu hỏi ``what-if'' để ChatGPT trả lời (ví dụ: ``Nếu RSI > 80 thì bạn khuyên gì?''), chuẩn bị demo cho buổi 4.
\end{itemize}

\section*{5. Ghi chú cho giáo viên}
\begin{itemize}[leftmargin=1.2cm]
    \item Chuẩn bị sẵn key OpenAI dự phòng vì học sinh dễ gặp lỗi quota.
    \item Nếu hạ tầng mạng yếu, cho phép HS dùng log text thay vì GUI.
    \item Nhắc HS ghi chép các prompt đã thử để so sánh hiệu quả ở buổi sau (hướng tới ``prompt engineering'').
\end{itemize}

\vspace{0.5cm}
\begin{center}
\textbf{Kết thúc buổi 3:}\\
\textit{Học sinh hiểu và có thể tinh chỉnh ChatGPT Advisor, theo dõi báo cáo tự động, và đánh giá hiệu suất giao dịch dựa trên dữ liệu thực nghiệm.}
\end{center}

\end{document}

