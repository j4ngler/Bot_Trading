% -*- coding: utf-8 -*-
\documentclass[12pt,a4paper]{article}
\usepackage[utf8]{inputenc}
\usepackage[vietnamese]{babel}
\usepackage{amsmath,amssymb}
\usepackage{geometry}
\usepackage{fancyhdr}
\usepackage{booktabs}
\usepackage{graphicx}
\usepackage{xcolor}
\usepackage{enumitem}

\geometry{margin=2.3cm}
\pagestyle{fancy}
\fancyhf{}
\fancyhead[L]{Buổi 4 - Quản lý rủi ro \& Vị thế}
\fancyhead[R]{Trading Bot - Testnet}
\fancyfoot[C]{\thepage}

\title{\textbf{Giáo Án Buổi 4}\\
\large Quản Lý Rủi Ro \& Vị Thế: Công Thức Tính Khối Lượng, SL/TP, R/R}
\author{}
\date{}

\begin{document}

\maketitle

\section*{1. Mục tiêu buổi học}
\begin{itemize}[leftmargin=1.2cm]
    \item Hiểu và vận dụng quy tắc 1\% vốn: chỉ rủi ro tối đa 1\% số dư mỗi lệnh.
    \item Nắm công thức tính khối lượng vị thế dựa trên stop loss và rủi ro chấp nhận được.
    \item Hiểu cách đặt Stop Loss và Take Profit dựa trên ATR (linh hoạt) hoặc phần trăm cố định.
    \item Tính được tỷ lệ Risk/Reward (R/R) và đánh giá xem lệnh có đáng thực hiện không (R/R $\geq$ 1.5).
    \item Thực hành đọc code trong \texttt{risk\_manager.py} để hiểu cách bot tính toán vị thế.
\end{itemize}

\section*{2. Chuẩn bị}
\begin{itemize}[leftmargin=1.2cm]
    \item Máy đã cài đặt môi trường, có thể chạy GUI và xem log.
    \item File \texttt{src/risk\_manager.py} và \texttt{src/config.py} đã mở sẵn trong editor.
    \item Máy tính cầm tay hoặc công cụ tính toán để thực hành công thức.
    \item Bảng ví dụ số liệu mẫu: Vốn = \$10,000, Giá BTC = \$43,250, ATR = \$250, Risk = 1\%.
\end{itemize}

\section*{3. Lộ trình 90 phút}

\subsection*{A. Warm-up: ``Nếu bạn có \$10,000, bạn dám rủi ro bao nhiêu?'' (5')}
\textbf{Hoạt động:} GV đặt câu hỏi mở:
\begin{itemize}[leftmargin=0.5cm]
    \item ``Nếu bạn có \$10,000 và muốn trade BTC, bạn sẽ mua bao nhiêu?''
    \item ``Nếu giá BTC giảm 5\%, bạn sẽ mất bao nhiêu?''
    \item ``Bạn có chấp nhận mất \$500 (5\%) trong một lệnh không?''
\end{itemize}
\textbf{Kết luận:} Hầu hết trader nghiệp dư không có quy tắc rõ ràng, dẫn đến rủi ro quá cao hoặc quá thấp. Bot sử dụng quy tắc 1\% để đảm bảo an toàn: chỉ rủi ro tối đa \$100 trên \$10,000 vốn mỗi lệnh.

\subsection*{B. Quy tắc 1\% vốn (15')}

\textbf{Bước 1: Giải thích quy tắc (5')}
\begin{itemize}[leftmargin=0.5cm]
    \item \textbf{Quy tắc 1\%:} Chỉ rủi ro tối đa 1\% số dư tài khoản mỗi lệnh.
    \item Ví dụ: Vốn = \$10,000 $\Rightarrow$ Rủi ro tối đa = \$10,000 $\times$ 1\% = \$100
    \item \textbf{Tại sao 1\%?}
    \begin{itemize}
        \item Nếu mất 10 lệnh liên tiếp, bạn chỉ mất 10\% vốn (vẫn còn 90\% để tiếp tục)
        \item Nếu rủi ro 5\% mỗi lệnh, chỉ cần 20 lệnh thua là mất hết vốn
        \item Quy tắc này giúp trader sống sót qua các chuỗi lệnh thua
    \end{itemize}
\end{itemize}

\textbf{Bước 2: Thực hành tính toán (10')}
\begin{itemize}[leftmargin=0.5cm]
    \item \textbf{Bài tập 1:} Vốn = \$10,000, Risk = 1\%. Tính số tiền rủi ro tối đa?
    \begin{itemize}
        \item Đáp án: \$10,000 $\times$ 0.01 = \$100
    \end{itemize}
    \item \textbf{Bài tập 2:} Vốn = \$5,000, Risk = 1.5\%. Tính số tiền rủi ro?
    \begin{itemize}
        \item Đáp án: \$5,000 $\times$ 0.015 = \$75
    \end{itemize}
    \item \textbf{Bài tập 3:} Nếu bạn có \$20,000 và muốn rủi ro 2\% mỗi lệnh, bạn sẽ mất bao nhiêu nếu thua 5 lệnh liên tiếp?
    \begin{itemize}
        \item Đáp án: \$20,000 $\times$ 0.02 $\times$ 5 = \$2,000 (mất 10\% vốn)
    \end{itemize}
\end{itemize}

\textit{Liên hệ kinh tế:} Quy tắc 1\% tương tự nguyên tắc ``đa dạng hóa rủi ro'' trong đầu tư: không bao giờ đặt tất cả trứng vào một giỏ. Trong trading, điều này có nghĩa là không bao giờ rủi ro quá nhiều trong một lệnh duy nhất.

\subsection*{C. Công thức tính khối lượng vị thế (25')}

\textbf{Bước 1: Giới thiệu công thức (10')}
\begin{itemize}[leftmargin=0.5cm]
    \item \textbf{Công thức cơ bản:}
    \[
    \text{Khối lượng} = \frac{\text{Vốn} \times \text{Risk\%}}{\text{Giá Entry} \times \text{SL\%}}
    \]
    \item \textbf{Giải thích từng thành phần:}
    \begin{itemize}
        \item \textbf{Vốn $\times$ Risk\%:} Số tiền bạn sẵn sàng mất (ví dụ: \$10,000 $\times$ 1\% = \$100)
        \item \textbf{Giá Entry $\times$ SL\%:} Khoảng cách stop loss tính bằng tiền (ví dụ: \$43,250 $\times$ 2\% = \$865)
        \item \textbf{Khối lượng:} Số lượng coin/token bạn sẽ mua để nếu giá giảm SL\%, bạn chỉ mất đúng số tiền rủi ro đã định
    \end{itemize}
    \item \textbf{Ví dụ cụ thể:}
    \begin{itemize}
        \item Vốn: \$10,000
        \item Risk: 1\% = \$100
        \item Giá BTC: \$43,250
        \item Stop Loss: 2\% = \$43,250 $\times$ 0.02 = \$865
        \item Khối lượng = \$\frac{100}{865} \approx 0.1156$ BTC
        \item Kiểm tra: Nếu giá giảm 2\% từ \$43,250 xuống \$42,393.5, bạn mất: 0.1156 $\times$ \$865 = \$100 ✓
    \end{itemize}
\end{itemize}

\textbf{Bước 2: Đọc code trong risk\_manager.py (10')}
\begin{itemize}[leftmargin=0.5cm]
    \item HS mở file \texttt{src/risk\_manager.py}, tìm hàm \texttt{calculate\_position\_size()}
    \item GV hướng dẫn đọc từng dòng:
    \begin{verbatim}
    risk_amount = self.account_balance * (self.risk_percent / 100)
    stop_loss_amount = entry_price * (self.stop_loss_percent / 100)
    quantity = risk_amount / stop_loss_amount
    \end{verbatim}
    \item So sánh với công thức trên: code đang làm đúng theo công thức!
    \item \textbf{Điểm đặc biệt:} Code còn xử lý trường hợp có ATR:
    \begin{itemize}
        \item Nếu có ATR, dùng \texttt{atr\_stop\_loss = current\_atr * 2} thay vì SL\% cố định
        \item Điều này linh hoạt hơn vì ATR phản ánh biến động thực tế của thị trường
    \end{itemize}
\end{itemize}

\textbf{Bước 3: Thực hành tính toán với số liệu thật (5')}
\begin{itemize}[leftmargin=0.5cm]
    \item Chạy bot, quan sát log khi bot tính vị thế
    \item So sánh số liệu trong log với công thức:
    \begin{itemize}
        \item Log hiển thị: ``💰 Tính toán vị thế: Khối lượng: 0.1156, Rủi ro: \$100.00''
        \item HS kiểm tra lại bằng công thức để xác nhận
    \end{itemize}
    \item \textbf{Bài tập:} Thay đổi \texttt{RISK\_PERCENTAGE} trong \texttt{config.py} từ 1.0 thành 2.0, chạy lại bot và quan sát khối lượng thay đổi như thế nào?
\end{itemize}

\subsection*{D. Stop Loss và Take Profit (20')}

\textbf{Bước 1: Stop Loss dựa trên ATR (10')}
\begin{itemize}[leftmargin=0.5cm]
    \item \textbf{Tại sao dùng ATR?}
    \begin{itemize}
        \item ATR đo biến động thực tế của thị trường
        \item Khi ATR cao (thị trường biến động mạnh), cần đặt SL xa hơn để tránh bị ``stop out'' sớm
        \item Khi ATR thấp (thị trường ổn định), có thể đặt SL gần hơn để bảo vệ vốn tốt hơn
    \end{itemize}
    \item \textbf{Công thức trong bot:}
    \begin{itemize}
        \item Stop Loss = Entry Price $\pm$ (2 $\times$ ATR) [Dấu + cho SELL, dấu - cho BUY]
        \item Take Profit = Entry Price $\pm$ (3 $\times$ ATR)
        \item Ví dụ: Entry = \$43,250, ATR = \$250
        \begin{itemize}
            \item SL (BUY) = \$43,250 - (2 $\times$ \$250) = \$42,750
            \item TP (BUY) = \$43,250 + (3 $\times$ \$250) = \$44,000
        \end{itemize}
    \end{itemize}
    \item \textbf{So sánh với SL\% cố định:}
    \begin{itemize}
        \item SL\% cố định: SL = \$43,250 $\times$ 0.98 = \$42,385 (cách entry 2\%)
        \item SL dựa ATR: SL = \$42,750 (cách entry 1.16\%)
        \item ATR linh hoạt hơn, phù hợp với điều kiện thị trường thực tế
    \end{itemize}
\end{itemize}

\textbf{Bước 2: Take Profit và Risk/Reward Ratio (10')}
\begin{itemize}[leftmargin=0.5cm]
    \item \textbf{Risk/Reward Ratio (R/R):}
    \[
    R/R = \frac{\text{Khoảng cách Take Profit}}{\text{Khoảng cách Stop Loss}}
    \]
    \item \textbf{Ví dụ:}
    \begin{itemize}
        \item Entry: \$43,250
        \item SL: \$42,750 (risk = \$500)
        \item TP: \$44,000 (reward = \$750)
        \item R/R = \$\frac{750}{500} = 1.5
    \end{itemize}
    \item \textbf{Ý nghĩa R/R:}
    \begin{itemize}
        \item R/R = 1.5 nghĩa là: mỗi \$1 rủi ro, bạn kỳ vọng kiếm \$1.5 lợi nhuận
        \item R/R $\geq$ 1.5 được coi là tốt (lợi nhuận tiềm năng lớn hơn rủi ro)
        \item R/R < 1.0 nghĩa là rủi ro lớn hơn lợi nhuận $\Rightarrow$ không nên vào lệnh
    \end{itemize}
    \item \textbf{Đọc code:} Tìm hàm \texttt{calculate\_risk\_reward\_ratio()} trong \texttt{risk\_manager.py}
\end{itemize}

\subsection*{E. Thực hành tổng hợp (20')}

\textbf{Bài tập lớn: Tính toán hoàn chỉnh một lệnh (15')}
\begin{itemize}[leftmargin=0.5cm]
    \item \textbf{Đề bài:}
    \begin{itemize}
        \item Vốn: \$10,000
        \item Risk: 1\%
        \item Giá BTC hiện tại: \$43,250
        \item ATR: \$250
        \item Signal: BUY
    \end{itemize}
    \item \textbf{Yêu cầu:}
    \begin{enumerate}
        \item Tính số tiền rủi ro tối đa
        \item Tính Stop Loss và Take Profit (dùng ATR)
        \item Tính khối lượng vị thế
        \item Tính Risk/Reward Ratio
        \item Kết luận: Có nên vào lệnh không? (R/R $\geq$ 1.5?)
    \end{enumerate}
    \item \textbf{Đáp án gợi ý:}
    \begin{itemize}
        \item Rủi ro = \$10,000 $\times$ 1\% = \$100
        \item SL = \$43,250 - (2 $\times$ \$250) = \$42,750
        \item TP = \$43,250 + (3 $\times$ \$250) = \$44,000
        \item Khối lượng = \$\frac{100}{500} = 0.2$ BTC
        \item R/R = \$\frac{750}{500} = 1.5 ✓ (Đáng vào lệnh)
    \end{itemize}
\end{itemize}

\textbf{So sánh với bot (5')}
\begin{itemize}[leftmargin=0.5cm]
    \item Chạy bot với số liệu tương tự, quan sát log
    \item So sánh kết quả tính toán của bot với kết quả tính tay
    \item Thảo luận: Có khác biệt không? Vì sao?
\end{itemize}

\subsection*{F. Tổng kết và Q\&A (5')}
\begin{itemize}[leftmargin=0.5cm]
    \item GV tổng hợp:
    \begin{itemize}
        \item Quy tắc 1\% giúp bảo vệ vốn
        \item Công thức tính khối lượng đảm bảo rủi ro không vượt quá ngưỡng
        \item SL/TP dựa trên ATR linh hoạt hơn SL\% cố định
        \item R/R $\geq$ 1.5 là tiêu chuẩn cho lệnh tốt
    \end{itemize}
    \item Q\&A: HS đặt câu hỏi về phần chưa hiểu
\end{itemize}

\section*{4. Bài tập về nhà}
\begin{itemize}[leftmargin=1.2cm]
    \item Tính toán hoàn chỉnh 2 lệnh mẫu khác (với số liệu khác nhau) và kiểm tra R/R.
    \item Đọc kỹ hàm \texttt{calculate\_position\_size()} trong \texttt{risk\_manager.py}, viết comment giải thích từng bước tính toán.
    \item (Tùy chọn) Thử thay đổi \texttt{STOP\_LOSS\_PERCENT} và \texttt{TAKE\_PROFIT\_PERCENT} trong \texttt{config.py}, quan sát sự thay đổi của R/R ratio.
\end{itemize}

\section*{5. Ghi chú cho giáo viên}
\begin{itemize}[leftmargin=1.2cm]
    \item Nhấn mạnh rằng quy tắc 1\% là ``kim chỉ nam'' trong quản lý rủi ro, không phải con số tuyệt đối.
    \item Khuyến khích HS tính tay trước khi xem kết quả của bot để hiểu sâu hơn.
    \item Nếu HS gặp khó khăn với công thức, có thể dùng ví dụ trực quan: ``Nếu bạn có 100 viên kẹo và chỉ muốn mất tối đa 1 viên mỗi lần chơi...''
    \item Nhắc nhở: Luôn kiểm tra R/R trước khi vào lệnh, đây là thói quen tốt của trader chuyên nghiệp.
\end{itemize}

\vspace{0.5cm}
\begin{center}
\textbf{Kết thúc buổi 4:}\\
\textit{Học sinh hiểu và có thể tính toán khối lượng vị thế, đặt SL/TP dựa trên ATR, và đánh giá lệnh qua R/R ratio.}
\end{center}

\end{document}
