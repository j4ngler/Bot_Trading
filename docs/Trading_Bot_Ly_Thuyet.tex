\documentclass[12pt,a4paper]{article}
\usepackage[utf8]{inputenc}
\usepackage[vietnamese]{babel}
\usepackage{amsmath,amssymb}
\usepackage{geometry}
\usepackage{fancyhdr}
\usepackage{booktabs}
\usepackage{longtable}
\usepackage{graphicx}
\usepackage{xcolor}
\usepackage{hyperref}
\usepackage{enumitem}

% Cấu hình trang
\geometry{margin=2.5cm}
\pagestyle{fancy}
\fancyhf{}
\fancyhead[L]{\leftmark}
\fancyhead[R]{Trading Bot - Lý Thuyết}
\fancyfoot[C]{\thepage}

% Màu sắc
\definecolor{primary}{RGB}{0,102,204}
\definecolor{secondary}{RGB}{51,153,51}
\definecolor{warning}{RGB}{255,153,0}

% Tiêu đề
\title{\textbf{Trading Bot - Trợ Lý Ảo Giao Dịch Tự Động}\\
\large Hệ thống giao dịch tự động sử dụng ChatGPT AI + Binance Testnet\\
\small Phù hợp cho học sinh cấp 3 - Học về AI, API, và Trading}
\author{}
\date{}

\begin{document}

\maketitle

\section{Tổng Quan}

\subsection{Mục Tiêu Dự Án}

Xây dựng một \textbf{trading bot tự động} có khả năng:

\begin{itemize}
    \item Thu thập dữ liệu thị trường từ Binance Testnet (an toàn)
    \item Tính toán chỉ báo kỹ thuật (MA, RSI, ATR)
    \item Sử dụng ChatGPT AI để phân tích và đưa ra khuyến nghị
    \item Tự động thực thi lệnh (Mua/Bán) dựa trên AI
    \item Logging và báo cáo hiệu suất giao dịch
\end{itemize}

\subsection{Luồng Hoạt Động}

Hệ thống hoạt động theo các bước sau:

\begin{enumerate}
    \item \textbf{Thu thập dữ liệu} $\rightarrow$ Lấy giá BTC/USDT từ Binance
    \item \textbf{Tính chỉ số} $\rightarrow$ MA, RSI, ATR
    \item \textbf{AI phân tích} $\rightarrow$ ChatGPT đưa ra khuyến nghị BUY/SELL/HOLD
    \item \textbf{Kiểm tra rủi ro} $\rightarrow$ Risk Manager xác định có an toàn giao dịch
    \item \textbf{Tính vị thế} $\rightarrow$ Tính khối lượng, stop loss, take profit
    \item \textbf{Thực thi} $\rightarrow$ Gửi lệnh (nếu hợp lý)
    \item \textbf{Log} $\rightarrow$ Lưu vào database
    \item \textbf{Báo cáo} $\rightarrow$ Tổng hợp hiệu suất và vẽ biểu đồ
\end{enumerate}

\section{Thuật Ngữ, Viết Tắt Và Ký Hiệu}

Tài liệu sử dụng nhiều thuật ngữ chuyên môn. Bảng dưới đây giúp người đọc nắm bắt nhanh:

\begin{description}[leftmargin=3cm,labelsep=0.6cm]
    \item[AI] Trí tuệ nhân tạo (Artificial Intelligence) -- công nghệ cho phép hệ thống học từ dữ liệu và đưa ra quyết định tự động.
    \item[API] Giao diện lập trình ứng dụng (Application Programming Interface) -- bộ quy tắc cho phép các hệ thống phần mềm giao tiếp và trao đổi dữ liệu.
    \item[ATR] Average True Range -- chỉ báo đo biên độ dao động trung bình của giá trong một giai đoạn nhất định.
    \item[ATR/giá] Tỷ lệ giữa ATR và giá hiện tại; dùng để đánh giá mức độ biến động so với giá trị đang giao dịch.
    \item[BTC/USDT] Cặp giao dịch giữa Bitcoin (BTC) và Tether (USDT) -- USDT là stablecoin neo theo đô la Mỹ.
    \item[BUY/SELL/HOLD] Ba trạng thái hành động của bot: mua vào, bán ra hoặc giữ nguyên vị thế hiện tại.
    \item[ChatGPT] Mô hình ngôn ngữ của OpenAI được sử dụng để phân tích dữ liệu và đưa ra khuyến nghị.
    \item[EMA] Exponential Moving Average -- đường trung bình động hàm mũ, nhấn mạnh dữ liệu giá gần nhất.
    \item[EMH] Efficient Market Hypothesis -- giả thuyết thị trường hiệu quả; giá phản ánh đầy đủ thông tin quá khứ.
    \item[FOMO] Fear Of Missing Out -- tâm lý sợ bỏ lỡ cơ hội, dễ dẫn đến mua đuổi ở đỉnh.
    \item[GUI] Graphical User Interface -- giao diện đồ họa với nút bấm, ô nhập, biểu đồ.
    \item[HOLD] Giữ nguyên vị thế, không mở thêm lệnh mới.
    \item[Log] Nhật ký lưu lại sự kiện và kết quả để phục vụ truy vết và đánh giá hiệu suất.
    \item[MA] Moving Average -- đường trung bình động phản ánh xu hướng giá.
    \item[Position Sizing] Quy trình tính khối lượng lệnh phù hợp với mức rủi ro cho phép.
    \item[R/R] Risk/Reward Ratio -- tỷ lệ giữa lợi nhuận kỳ vọng và rủi ro chấp nhận.
    \item[Risk\%] Phần trăm vốn tối đa chấp nhận rủi ro trên mỗi lệnh.
    \item[RSI] Relative Strength Index -- chỉ báo đo sức mạnh tương đối của xu hướng giá.
    \item[SL] Stop Loss -- mức giá cắt lỗ tự động để giới hạn thua lỗ.
    \item[TP] Take Profit -- mức giá chốt lời tự động để khóa lợi nhuận.
    \item[StopLoss\% / TakeProfit\%] Mức phần trăm cố định dùng để đặt SL/TP khi không có dữ liệu ATR.
    \item[Loss Aversion] Thiên kiến ghét thua lỗ -- khuynh hướng giữ lệnh lỗ quá lâu vì kỳ vọng hồi phục.
    \item[$\rightarrow$] Ký hiệu mũi tên thể hiện luồng xử lý hoặc quan hệ nguyên nhân--kết quả giữa các bước.
\end{description}

\section{Khái Niệm Chỉ Báo Kỹ Thuật}

\subsection{RSI (Relative Strength Index) - Chỉ Số Sức Mạnh Tương Đối}

\subsubsection{Định nghĩa}

RSI là chỉ báo đo tốc độ và quy mô biến động giá gần đây. Giá trị từ 0 đến 100, được phát triển bởi J. Welles Wilder (1978).

\subsubsection{Công thức}

\begin{equation}
\text{RSI} = 100 - \frac{100}{1 + \text{RS}}
\end{equation}

Trong đó: $\text{RS} = \frac{\text{Trung bình tăng}}{\text{Trung bình giảm}}$ (trong 14 phiên)

\subsubsection{Ý nghĩa}

\begin{itemize}
    \item \textbf{RSI > 70}: Thị trường \textbf{QUÁ MUA} (Overbought)
    \begin{itemize}
        \item Nhiều người đã mua $\rightarrow$ Cầu giảm $\rightarrow$ Giá có thể điều chỉnh xuống
        \item Bot không mua khi RSI > 70
    \end{itemize}
    
    \item \textbf{RSI < 30}: Thị trường \textbf{QUÁ BÁN} (Oversold)
    \begin{itemize}
        \item Nhiều người đã bán $\rightarrow$ Cung giảm $\rightarrow$ Giá có thể phục hồi
        \item Bot cân nhắc mua khi RSI < 30
    \end{itemize}
    
    \item \textbf{RSI 30-70}: Thị trường \textbf{CÂN BẰNG}
    \begin{itemize}
        \item Không có tín hiệu rõ ràng $\rightarrow$ Bot thường HOLD
    \end{itemize}
\end{itemize}

\subsubsection{Ví dụ thực tế}

Giá BTC: \$43,250\\
RSI: 72.5\\
$\rightarrow$ RSI > 70 $\rightarrow$ QUÁ MUA\\
$\rightarrow$ Bot khuyến nghị: SELL hoặc HOLD\\
$\rightarrow$ Không nên mua vào lúc này

\subsection{MA (Moving Average) - Đường Trung Bình Động}

\subsubsection{Định nghĩa}

MA là giá trung bình của một tài sản trong N phiên gần nhất. Làm mịn biến động giá, giúp nhận diện xu hướng.

\subsubsection{Các loại MA}

\begin{itemize}
    \item \textbf{SMA (Simple Moving Average)}: Trung bình số học đơn giản
    \begin{equation}
    \text{SMA}(20) = \frac{\text{Giá}_1 + \text{Giá}_2 + \cdots + \text{Giá}_{20}}{20}
    \end{equation}
    
    \item \textbf{EMA (Exponential Moving Average)}: Trung bình hàm mũ
    \begin{itemize}
        \item Ưu tiên dữ liệu gần đây hơn
        \item Phản ứng nhanh hơn với biến động mới
    \end{itemize}
\end{itemize}

\subsubsection{Ý nghĩa}

\begin{itemize}
    \item \textbf{Giá > MA}: Xu hướng \textbf{TĂNG} (Bullish)
    \begin{itemize}
        \item Giá đang ở trên mức trung bình $\rightarrow$ Lực mua mạnh
        \item Bot có thể cân nhắc BUY
    \end{itemize}
    
    \item \textbf{Giá < MA}: Xu hướng \textbf{GIẢM} (Bearish)
    \begin{itemize}
        \item Giá đang ở dưới mức trung bình $\rightarrow$ Lực bán mạnh
        \item Bot có thể cân nhắc SELL
    \end{itemize}
    
    \item \textbf{MA ngắn cắt lên MA dài}: Tín hiệu \textbf{MUA} (Golden Cross)
    \item \textbf{MA ngắn cắt xuống MA dài}: Tín hiệu \textbf{BÁN} (Death Cross)
\end{itemize}

\subsubsection{Ví dụ thực tế}

Giá BTC: \$43,250\\
MA(20): \$42,800\\
$\rightarrow$ Giá > MA $\rightarrow$ Xu hướng TĂNG\\
$\rightarrow$ Bot phân tích: Thị trường đang tích cực

\subsection{ATR (Average True Range) - Biên Độ Dao Động Trung Bình}

\subsubsection{Định nghĩa}

ATR đo lường \textbf{mức độ biến động} (volatility) của giá. Được phát triển bởi J. Welles Wilder (1978). Giá trị càng cao $\rightarrow$ Thị trường càng biến động.

\subsubsection{Công thức}

True Range (TR) = Max của:
\begin{itemize}
    \item High - Low (biên độ trong phiên)
    \item $|\text{High} - \text{Close trước}|$ (gap tăng)
    \item $|\text{Low} - \text{Close trước}|$ (gap giảm)
\end{itemize}

\begin{equation}
\text{ATR} = \text{Trung bình của TR trong 14 phiên}
\end{equation}

\subsubsection{Ý nghĩa}

\begin{itemize}
    \item \textbf{ATR cao}: Thị trường \textbf{biến động mạnh}
    \begin{itemize}
        \item Giá nhảy mạnh $\rightarrow$ Cần đặt Stop Loss xa hơn
        \item Rủi ro cao $\rightarrow$ Bot cẩn thận hơn
    \end{itemize}
    
    \item \textbf{ATR thấp}: Thị trường \textbf{ổn định}
    \begin{itemize}
        \item Giá ít biến động $\rightarrow$ Có thể đặt Stop Loss gần hơn
        \item Rủi ro thấp $\rightarrow$ Bot có thể giao dịch an toàn hơn
    \end{itemize}
\end{itemize}

\subsubsection{Ứng dụng trong bot}

Đặt Stop Loss động:

\begin{itemize}
    \item Nếu có ATR:
    \begin{itemize}
        \item Stop Loss = Entry Price - (2 $\times$ ATR) [cho lệnh BUY]
        \item Take Profit = Entry Price + (3 $\times$ ATR)
    \end{itemize}
    
    \item Nếu không có ATR:
    \begin{itemize}
        \item Stop Loss = Entry Price $\times$ (1 - 2\%) [cố định 2\%]
        \item Take Profit = Entry Price $\times$ (1 + 3\%) [cố định 3\%]
    \end{itemize}
\end{itemize}

\subsubsection{Ví dụ thực tế}

Giá BTC: \$43,250\\
ATR: \$250\\
$\rightarrow$ ATR = \$250 $\rightarrow$ Biến động trung bình

\textbf{Stop Loss (dùng ATR):}
\begin{align}
&= \$43,250 - (2 \times \$250)\\
&= \$42,750 \quad \text{[Cách entry 1.16\%]}
\end{align}

\textbf{Stop Loss (cố định 2\%):}
\begin{align}
&= \$43,250 \times 0.98\\
&= \$42,385 \quad \text{[Cách entry 2\%]}
\end{align}

$\rightarrow$ Dùng ATR linh hoạt hơn, phù hợp với biến động thực tế

\subsection{Tổng Hợp: Cách Bot Sử Dụng 3 Chỉ Báo}

\begin{longtable}{p{3cm}p{4cm}p{6cm}}
\toprule
\textbf{Chỉ Báo} & \textbf{Mục Đích} & \textbf{Bot Sử Dụng} \\
\midrule
\textbf{RSI} & Phát hiện quá mua/quá bán & Chặn giao dịch khi RSI > 70 hoặc < 30 \\
\midrule
\textbf{MA} & Xác định xu hướng & So sánh giá với MA để quyết định BUY/SELL \\
\midrule
\textbf{ATR} & Đo biến động & Điều chỉnh Stop Loss/Take Profit linh hoạt \\
\bottomrule
\end{longtable}

\subsubsection{Ví dụ kết hợp}

\begin{itemize}
    \item Giá: \$43,250
    \item MA(20): \$42,800 $\rightarrow$ Giá > MA (xu hướng tăng)
    \item RSI: 65 $\rightarrow$ Cân bằng (30-70)
    \item ATR: \$250 $\rightarrow$ Biến động bình thường
\end{itemize}

$\rightarrow$ Bot phân tích:
\begin{itemize}
    \item Xu hướng tăng (giá > MA)
    \item RSI an toàn (không quá cực)
    \item ATR ổn định
\end{itemize}

$\rightarrow$ ChatGPT có thể khuyến nghị: BUY\\
$\rightarrow$ Risk Manager kiểm tra: Pass\\
$\rightarrow$ Bot thực thi lệnh với Stop Loss = 2$\times$ATR

\section{Lý Thuyết Kinh Tế – Tài Chính Nền Tảng}

\subsection{Kinh tế học vi mô: Cung – Cầu và Kỳ vọng}

\begin{itemize}
    \item \textbf{Cung – Cầu}: Giá tăng khi cầu > cung; giảm khi cung > cầu. Trong crypto, kỳ vọng tương lai làm cầu thay đổi rất nhanh.
    
    \item \textbf{Ứng dụng trong bot}:
    \begin{itemize}
        \item RSI > 70 hiểu như trạng thái ``đã có quá nhiều người mua'' $\rightarrow$ cầu suy yếu $\rightarrow$ rủi ro đảo chiều tăng $\rightarrow$ Bot không giao dịch.
        \item RSI < 30 hiểu như ``đã có quá nhiều người bán'' $\rightarrow$ áp lực cung suy yếu $\rightarrow$ dễ phục hồi, nhưng bot vẫn cần kiểm tra điều kiện khác (confidence $\geq$ 50\%, ATR/giá $\leq$ 25\%).
    \end{itemize}
    
    \item \textbf{Kỳ vọng \& EMH (Efficient Market Hypothesis)}: Bot phản ứng theo dữ liệu gần nhất (RSI/MA/ATR) tương ứng giả định thị trường hiệu quả mức ``yếu'' (giá phản ánh dữ liệu quá khứ), nên vẫn còn chỗ cho chiến lược phản ứng nhanh.
\end{itemize}

\subsection{Kinh tế học hành vi: Vì sao cần kỷ luật máy móc}

\textbf{Tại sao con người thường thua lỗ trong trading?} Nghiên cứu cho thấy 90\% trader thua lỗ không phải vì thiếu kiến thức, mà vì \textbf{cảm xúc chi phối quyết định}. Bot tự động loại bỏ yếu tố này.

\subsubsection{Loss Aversion (Ghét thua lỗ - Thiên kiến mất mát)}

\textbf{Định nghĩa:} Con người cảm nhận nỗi đau mất \$100 mạnh gấp 2-2.5 lần niềm vui khi kiếm được \$100. Điều này khiến trader:
\begin{itemize}
    \item Giữ lệnh lỗ quá lâu, hy vọng giá quay lại
    \item Cắt lời quá sớm vì sợ mất lợi nhuận đã có
    \item Không dám vào lệnh mới sau khi thua
\end{itemize}

\textbf{Ví dụ thực tế:}
\begin{itemize}
    \item Trader mua BTC ở \$40,000
    \item Giá giảm xuống \$38,000 (lỗ \$2,000)
    \item ``Chờ giá quay lại, không bán lỗ!''
    \item Giá tiếp tục giảm xuống \$35,000 (lỗ \$5,000)
    \item Vẫn không bán vì ``đã lỗ rồi, chờ thêm''
    \item Cuối cùng giá xuống \$30,000 $\rightarrow$ Mất \$10,000
\end{itemize}

\textbf{Bot xử lý như thế nào:}
\begin{itemize}
    \item Bot đặt \textbf{Stop Loss cứng} ngay khi vào lệnh (không có cảm xúc)
    \item Nếu có ATR: SL = Entry - (2 $\times$ ATR)
    \item Nếu không: SL = Entry $\times$ (1 - 2\%)
    \item Khi giá chạm Stop Loss $\rightarrow$ Bot tự động bán, không do dự
    \item \textbf{Kết quả:} Chỉ mất tối đa 1\% vốn/lệnh, không bao giờ ``thổi bay'' tài khoản
\end{itemize}

\subsubsection{FOMO (Fear Of Missing Out - Sợ bỏ lỡ)}

\textbf{Định nghĩa:} Khi thấy giá tăng mạnh, trader sợ bỏ lỡ cơ hội $\rightarrow$ mua đuổi ở đỉnh, thường là lúc giá sắp đảo chiều.

\textbf{Ví dụ thực tế:}
\begin{itemize}
    \item BTC đang ở \$40,000
    \item Giá tăng lên \$42,000 (tin tức tốt)
    \item Trader: ``Mình đã bỏ lỡ! Phải mua ngay!''
    \item Mua ở \$42,000
    \item Giá đảo chiều, giảm xuống \$38,000
    \item Lỗ \$4,000 ngay lập tức
\end{itemize}

\textbf{Bot xử lý như thế nào:}
\begin{itemize}
    \item Bot kiểm tra điều kiện:
    \begin{itemize}
        \item Chặn giao dịch nếu RSI > 70 (quá mua)
        \item Chặn nếu biến động quá cao: ATR/giá > 25\%
        \item Chặn nếu độ tin cậy của AI < 50\% (confidence threshold)
    \end{itemize}
    \item Khi RSI > 70 hoặc ATR/giá > 25\% hoặc confidence < 50\% $\rightarrow$ Bot \textbf{tự động từ chối} giao dịch
    \item \textbf{Kết quả:} Bot không bao giờ mua đuổi ở đỉnh, tránh được FOMO
\end{itemize}

\subsubsection{Herding (Bầy đàn - Tâm lý đám đông)}

\textbf{Định nghĩa:} Con người có xu hướng làm theo đám đông. Khi nhiều người mua $\rightarrow$ giá tăng $\rightarrow$ nhiều người mua hơn $\rightarrow$ tạo bong bóng. Khi nhiều người bán $\rightarrow$ giá giảm $\rightarrow$ nhiều người bán hơn $\rightarrow$ tạo hoảng loạn.

\textbf{Ví dụ thực tế:}
\begin{itemize}
    \item Thị trường đang ``hưng phấn'':
    \begin{itemize}
        \item RSI = 78 (quá mua)
        \item Giá tăng 10\% trong 1 ngày
        \item Mọi người đều mua $\rightarrow$ FOMO lan truyền
    \end{itemize}
    \item Đây là lúc nguy hiểm nhất, giá sắp đảo chiều
\end{itemize}

\textbf{Bot xử lý như thế nào:}
\begin{itemize}
    \item Bot nhận diện ``bầy đàn'' qua:
    \begin{itemize}
        \item RSI cực cao (>70) $\rightarrow$ Nhiều người đã mua
        \item Biến động cao (ATR/giá > 25\%) $\rightarrow$ Thị trường không ổn định
    \end{itemize}
    \item Bot \textbf{không giao dịch} khi phát hiện tín hiệu này
    \item \textbf{Kết quả:} Bot tránh được ``bẫy đám đông'', giao dịch ngược lại khi thị trường quá cực
\end{itemize}

\subsubsection{Anchoring (Neo giá mua - Thiên kiến neo)}

\textbf{Định nghĩa:} Trader ``neo'' vào giá mua ban đầu, không chịu thay đổi quyết định dựa trên dữ liệu mới.

\textbf{Ví dụ thực tế:}
\begin{itemize}
    \item Trader mua BTC ở \$40,000
    \item Giá giảm xuống \$35,000
    \item Trader: ``Mình mua ở \$40k, phải đợi giá về \$40k mới bán''
    \item Bỏ qua tín hiệu bán từ chỉ báo kỹ thuật
    \item Giá tiếp tục giảm $\rightarrow$ Lỗ lớn
\end{itemize}

\textbf{Bot xử lý như thế nào:}
\begin{itemize}
    \item Bot \textbf{không nhớ} giá mua cũ
    \item Mỗi chu kỳ, bot phân tích lại từ đầu dựa trên:
    \begin{itemize}
        \item Giá hiện tại
        \item Chỉ báo kỹ thuật mới nhất (MA, RSI, ATR)
        \item Khuyến nghị ChatGPT mới nhất
    \end{itemize}
    \item Bot ra quyết định \textbf{hoàn toàn dựa trên dữ liệu hiện tại}, không bị ảnh hưởng bởi lịch sử
    \item \textbf{Kết quả:} Bot linh hoạt, thích ứng nhanh với thị trường
\end{itemize}

\subsection{Lý thuyết danh mục \& rủi ro: Vì sao chỉ mạo hiểm 1\% vốn/lệnh}

\subsubsection{Quy tắc 1\% - Bảo vệ tài khoản}

\textbf{Tại sao 1\%?} Nếu bạn mạo hiểm quá nhiều mỗi lệnh, chỉ cần 10-20 lệnh lỗ liên tiếp là tài khoản sẽ ``thổi bay''.

\begin{longtable}{c|c|c}
\toprule
\textbf{Rủi ro/lệnh} & \textbf{Số lệnh lỗ để mất 50\% vốn} & \textbf{Số lệnh lỗ để mất 100\% vốn} \\
\midrule
\textbf{1\%} (Bot dùng) & 69 lệnh & 100 lệnh \\
\midrule
5\% & 14 lệnh & 20 lệnh \\
\midrule
10\% & 7 lệnh & 10 lệnh \\
\midrule
20\% & 3 lệnh & 5 lệnh \\
\bottomrule
\end{longtable}

\textbf{Kết luận:} Với 1\% rủi ro, bạn có thể chịu được \textbf{100 lệnh lỗ liên tiếp} trước khi mất hết vốn. Điều này gần như không thể xảy ra nếu bot hoạt động đúng.

\subsubsection{Position Sizing (Tính khối lượng vị thế)}

\textbf{Công thức:}
\begin{equation}
\text{Khối lượng} = \frac{\text{Vốn} \times \text{Risk\%}}{\text{Giá} \times \text{StopLoss\%}}
\end{equation}

\textbf{Ví dụ cụ thể:}
\begin{itemize}
    \item Vốn: \$10,000
    \item Risk: 1\% = \$100
    \item Giá BTC: \$43,250
    \item Stop Loss: 2\% = \$865 (khoảng cách từ entry)
\end{itemize}

\begin{align}
\text{Khối lượng} &= \frac{\$100}{\$865} = 0.1156 \text{ BTC}\\
\text{Giá trị lệnh} &= 0.1156 \times \$43,250 = \$5,000
\end{align}

\textbf{Tại sao công thức này?}
\begin{itemize}
    \item Nếu giá giảm 2\% (chạm Stop Loss) $\rightarrow$ Bạn chỉ mất đúng \$100 (1\% vốn)
    \item Không phụ thuộc vào giá BTC $\rightarrow$ Luôn rủi ro 1\% dù BTC ở \$30k hay \$60k
\end{itemize}

\subsubsection{Risk/Reward Ratio (Tỷ lệ Rủi ro/Lợi nhuận)}

\textbf{Định nghĩa:} Tỷ lệ giữa lợi nhuận kỳ vọng và rủi ro tối đa.

\textbf{Công thức:}
\begin{equation}
\text{R/R} = \frac{\text{Take Profit} - \text{Entry}}{\text{Entry} - \text{Stop Loss}}
\end{equation}

\textbf{Ví dụ:}
\begin{itemize}
    \item Entry: \$43,250
    \item Stop Loss: \$42,385 (giảm 2\%)
    \item Take Profit: \$44,548 (tăng 3\%)
\end{itemize}

\begin{align}
\text{R/R} &= \frac{\$44,548 - \$43,250}{\$43,250 - \$42,385}\\
&= \frac{\$1,298}{\$865} = 1.5
\end{align}

\textbf{Ý nghĩa:}
\begin{itemize}
    \item R/R = 1.5 $\rightarrow$ Nếu thắng, bạn kiếm \$1.5 cho mỗi \$1 rủi ro
    \item Nếu tỷ lệ thắng 50\%, bạn vẫn có lời về lâu dài
    \item \textbf{Quy tắc:} Chỉ vào lệnh nếu R/R $\geq$ 1.5
\end{itemize}

\subsubsection{Tính số dư tài khoản thực tế}

\textbf{Công thức:}
\begin{equation}
\text{Account Balance} = \text{USDT Balance} + (\text{BTC Balance} \times \text{Current BTC Price})
\end{equation}

\textbf{Ví dụ:}
\begin{itemize}
    \item USDT trong ví: \$5,000
    \item BTC trong ví: 0.1 BTC
    \item Giá BTC hiện tại: \$40,000
\end{itemize}

\begin{align}
\text{Account Balance} &= \$5,000 + (0.1 \times \$40,000)\\
&= \$5,000 + \$4,000 = \$9,000
\end{align}

\textbf{Ý nghĩa:}
\begin{itemize}
    \item Phản ánh giá trị thực tế của tài khoản tại thời điểm hiện tại
    \item Bao gồm cả tiền mặt (USDT) và tài sản crypto (BTC)
    \item Được cập nhật từ Binance API sau mỗi chu kỳ
    \item Nếu không lấy được từ API, bot sẽ tính từ PnL: $\text{Balance} = \text{Initial Balance} + \text{Total PnL}$
\end{itemize}

\subsubsection{Realized PnL vs Unrealized PnL}

\textbf{Realized PnL (Lợi nhuận đã thực hiện):}
\begin{itemize}
    \item Lợi nhuận từ các lệnh đã đóng hoàn toàn
    \item Đã được ``khóa'' và không thể thay đổi
    \item Công thức: $\text{PnL} = (\text{Exit Price} - \text{Entry Price}) \times \text{Quantity}$
\end{itemize}

\textbf{Unrealized PnL (Lợi nhuận chưa thực hiện):}
\begin{itemize}
    \item Lợi nhuận tiềm năng từ các lệnh đang mở
    \item Tính theo giá hiện tại, có thể thay đổi
    \item Công thức: $\text{Unrealized} = (\text{Current Price} - \text{Entry Price}) \times \text{Quantity}$
\end{itemize}

\textbf{Ví dụ:}
\begin{itemize}
    \item Đã mua BTC ở \$40,000, đã bán ở \$42,000 $\rightarrow$ Realized PnL = +\$2,000
    \item Đang giữ BTC mua ở \$40,000, giá hiện tại \$41,000 $\rightarrow$ Unrealized PnL = +\$1,000
    \item Nếu giá giảm xuống \$39,000 $\rightarrow$ Unrealized PnL = -\$1,000
\end{itemize}

\textbf{Total PnL:}
\begin{equation}
\text{Total PnL} = \text{Realized PnL} + \text{Unrealized PnL}
\end{equation}

\textbf{Ý nghĩa trong bot:}
\begin{itemize}
    \item Bot tính toán cả Realized và Unrealized PnL để đánh giá hiệu suất tổng thể
    \item Realized PnL phản ánh lợi nhuận thực tế đã đạt được
    \item Unrealized PnL cho biết giá trị hiện tại của các vị thế đang mở
    \item Cả hai được hiển thị trong báo cáo hiệu suất
\end{itemize}

\subsubsection{Kelly Criterion (Tiêu chí Kelly - Nâng cao)}

\textbf{Định nghĩa:} Công thức toán học tính tỷ lệ vốn tối ưu dựa trên xác suất thắng và R/R ratio.

\textbf{Công thức:}
\begin{equation}
\text{Kelly \%} = \frac{\text{Win Rate} \times \text{R/R} - \text{Loss Rate}}{\text{R/R}}
\end{equation}

\textbf{Ví dụ:}
\begin{itemize}
    \item Win Rate: 60\% (thắng 6/10 lệnh)
    \item Loss Rate: 40\% (thua 4/10 lệnh)
    \item R/R: 1.5
\end{itemize}

\begin{align}
\text{Kelly \%} &= \frac{0.6 \times 1.5 - 0.4}{1.5}\\
&= \frac{0.9 - 0.4}{1.5}\\
&= \frac{0.5}{1.5} = 33.3\%
\end{align}

\textbf{Lưu ý:} Kelly Criterion thường quá mạo hiểm. Bot dùng \textbf{``Half Kelly''} (50\% của Kelly) hoặc \textbf{1\% cố định} để an toàn hơn.

\subsection{Volatility (ATR) và kịch bản dừng lỗ/chốt lời}

\subsubsection{ATR (Average True Range) - Đo biến động}

\textbf{Tại sao cần ATR?} Thị trường không phải lúc nào cũng biến động như nhau. Có ngày giá nhảy \$500, có ngày chỉ nhảy \$50. Stop Loss cố định 2\% không phù hợp với mọi tình huống.

\begin{longtable}{p{3.5cm}|c|c|p{2cm}|p{3cm}|p{3cm}}
\toprule
\textbf{Tình huống} & \textbf{Giá BTC} & \textbf{ATR} & \textbf{Biến động} & \textbf{Stop Loss cố định 2\%} & \textbf{Stop Loss dùng ATR} \\
\midrule
Thị trường ổn định & \$43,250 & \$100 & Thấp & \$865 (2\%) & \$200 (2$\times$ATR = 0.46\%) \\
\midrule
Thị trường biến động & \$43,250 & \$500 & Cao & \$865 (2\%) & \$1,000 (2$\times$ATR = 2.3\%) \\
\bottomrule
\end{longtable}

\textbf{Kết luận:}
\begin{itemize}
    \item Khi ATR thấp $\rightarrow$ Stop Loss gần hơn $\rightarrow$ Bảo vệ tốt hơn
    \item Khi ATR cao $\rightarrow$ Stop Loss xa hơn $\rightarrow$ Tránh bị ``stop out'' bởi noise
\end{itemize}

\textbf{Ví dụ cụ thể:}
\begin{itemize}
    \item Entry: \$43,250
    \item ATR: \$250
\end{itemize}

\begin{align}
\text{Stop Loss (dùng ATR)} &= \$43,250 - (2 \times \$250) = \$42,750\\
\text{Take Profit (dùng ATR)} &= \$43,250 + (3 \times \$250) = \$44,000
\end{align}

\begin{align}
\text{R/R} &= \frac{\$44,000 - \$43,250}{\$43,250 - \$42,750}\\
&= \frac{\$750}{\$500} = 1.5 \quad \checkmark \text{ (Hợp lý)}
\end{align}

\subsection{Chu kỳ thị trường và tâm lý số đông}

\subsubsection{Chu kỳ tâm lý thị trường}

Thị trường crypto trải qua các giai đoạn tâm lý lặp đi lặp lại:

\begin{enumerate}
    \item \textbf{Tích luỹ (Accumulation)}
    \begin{itemize}
        \item RSI thấp (<30), giá ổn định
        \item Ít người quan tâm
        \item Bot: Có thể mua (cơ hội tốt) - nhưng cần kiểm tra điều kiện khác (confidence $\geq$ 50\%, ATR/giá $\leq$ 25\%)
    \end{itemize}
    
    \item \textbf{Hoài nghi (Disbelief)}
    \begin{itemize}
        \item Giá bắt đầu tăng nhẹ
        \item Nhiều người vẫn hoài nghi
        \item Bot: Theo dõi, chờ tín hiệu rõ ràng
    \end{itemize}
    
    \item \textbf{Lạc quan (Optimism)}
    \begin{itemize}
        \item Giá tăng mạnh
        \item RSI tăng (50-70)
        \item Bot: Có thể mua (xu hướng tăng)
    \end{itemize}
    
    \item \textbf{Hưng phấn (Euphoria)} \textcolor{warning}{⚠️}
    \begin{itemize}
        \item RSI rất cao (>70)
        \item Giá tăng mạnh, mọi người FOMO
        \item Bot: KHÔNG GIAO DỊCH (quá nguy hiểm)
    \end{itemize}
    
    \item \textbf{Hoảng loạn (Panic)}
    \begin{itemize}
        \item Giá giảm mạnh
        \item RSI rất thấp (<30)
        \item Bot: KHÔNG GIAO DỊCH ngay (cần chờ tín hiệu rõ ràng hơn, kiểm tra confidence và ATR)
    \end{itemize}
    
    \item \textbf{Trầm cảm (Depression)}
    \begin{itemize}
        \item Giá tiếp tục giảm
        \item Nhiều người bán tháo
        \item Bot: Chờ tín hiệu tích luỹ
    \end{itemize}
\end{enumerate}

\textbf{Bot nhận diện chu kỳ như thế nào:}
\begin{itemize}
    \item Nếu RSI > 70: ``RSI quá cao - Thị trường quá mua (Hưng phấn)'' $\rightarrow$ Không giao dịch
    \item Nếu RSI < 30: ``RSI quá thấp - Thị trường quá bán (Hoảng loạn)'' $\rightarrow$ Không giao dịch
\end{itemize}

\textbf{Ví dụ thực tế:}
\begin{itemize}
    \item Tình huống: BTC tăng từ \$40k $\rightarrow$ \$50k trong 1 tuần
    \item RSI: 72 (quá mua)
    \item ATR: \$800 (biến động cao)
    \item $\rightarrow$ Bot nhận diện: ``Hưng phấn'' - Giai đoạn nguy hiểm
    \item $\rightarrow$ Bot: KHÔNG GIAO DỊCH
    \item $\rightarrow$ Kết quả: Giá đảo chiều, giảm về \$42k
    \item $\rightarrow$ Bot tránh được lỗ lớn
\end{itemize}

\subsection{Bảng quy chiếu nhanh: Lý thuyết $\rightarrow$ Thực thi trong bot}

\begin{longtable}{p{3cm}|p{5cm}|p{4cm}}
\toprule
\textbf{Lý thuyết} & \textbf{Ứng dụng trong bot} & \textbf{Ví dụ} \\
\midrule
\textbf{Cung–Cầu} & Dùng RSI/MA để suy luận áp lực mua bán & RSI > 70 $\rightarrow$ Cầu yếu $\rightarrow$ Không mua \\
\midrule
\textbf{Loss Aversion} & Stop Loss cứng, tự động cắt lỗ & SL = Entry - 2$\times$ATR \\
\midrule
\textbf{FOMO} & Chặn giao dịch khi RSI > 70 hoặc ATR/giá > 25\% & RSI = 72 $\rightarrow$ Không giao dịch \\
\midrule
\textbf{Herding} & Nhận diện cực trị qua RSI + ATR & RSI > 70 + ATR/giá > 25\% $\rightarrow$ Bỏ qua \\
\midrule
\textbf{Anchoring} & Mỗi chu kỳ phân tích lại từ đầu & Không nhớ giá cũ \\
\midrule
\textbf{Quy tắc 1\%} & Risk 1\%/lệnh, tính khối lượng theo SL & \$10k vốn $\rightarrow$ \$100 rủi ro/lệnh \\
\midrule
\textbf{Position Sizing} & Công thức: $\frac{\text{Vốn} \times 1\%}{\text{Giá} \times \text{SL\%}}$ & Quantity = \$100 / \$865 \\
\midrule
\textbf{Risk/Reward} & Tính R/R ratio, chỉ vào lệnh nếu $\geq$ 1.5 & R/R = 1.5 $\rightarrow$ OK \\
\midrule
\textbf{Volatility (ATR)} & Điều chỉnh SL/TP linh hoạt theo ATR & SL = 2$\times$ATR, TP = 3$\times$ATR \\
\midrule
\textbf{Chu kỳ tâm lý} & Nhận diện vùng cực trị (RSI > 70 hoặc < 30) & RSI = 72 $\rightarrow$ ``Hưng phấn'' $\rightarrow$ Không giao dịch \\
\bottomrule
\end{longtable}

\section{Cách Sử Dụng}

\subsection{Các Chế Độ Hoạt Động}

Sau khi khởi động hệ thống, bạn sẽ thấy menu chọn:

\begin{itemize}
    \item \textbf{Chế độ 0: Chạy với GIAO DIỆN GUI} - Giao diện đồ họa với các nút điều khiển
    \item \textbf{Chế độ 1: Chạy MỘT LẦN} - Phân tích và dừng
    \item \textbf{Chế độ 2: Chạy LIÊN TỤC} - Tự động mỗi 5 phút
    \item \textbf{Chế độ 3: Chạy DEMO} - Chỉ phân tích, không giao dịch
    \item \textbf{Chế độ 4: Xem BÁO CÁO} - Hiệu suất giao dịch (HTML + biểu đồ)
\end{itemize}

\subsection{Ví Dụ Output}

\begin{itemize}
    \item Giá: \$43,250
    \item MA: \$42,800
    \item RSI: 72.5
    \item ATR: \$250
    \item ChatGPT: SELL (RSI cao, quá mua)
    \item KHÔNG thực thi - RSI quá cực, không an toàn
\end{itemize}

\section{Lưu Ý Quan Trọng}

\begin{enumerate}
    \item \textbf{CHỈ DÙNG BINANCE TESTNET} - Không dùng tiền thật
    \item \textbf{API có chi phí} - OpenAI charge theo token
    \item \textbf{Không phải lời khuyên đầu tư} - Chỉ học tập
    \item \textbf{Rủi ro cao} - Trading có thể mất tiền
    \item \textbf{Backup code} - Commit thường xuyên
\end{enumerate}

\section{Tính Năng}

\begin{itemize}
    \item Binance Testnet integration
    \item Tính chỉ báo kỹ thuật (MA, RSI, ATR)
    \item ChatGPT AI phân tích thị trường
    \item Quản lý rủi ro tự động (Risk Manager)
    \begin{itemize}
        \item RSI threshold: > 70 (quá mua) hoặc < 30 (quá bán)
        \item Confidence threshold: $\geq$ 50\%
        \item ATR volatility threshold: > 25\% giá
    \end{itemize}
    \item Database và logging chi tiết
    \item Báo cáo hiệu suất và biểu đồ vốn
    \begin{itemize}
        \item Equity curve với X-axis là chu kỳ (cycle)
        \item Tính Realized và Unrealized PnL
        \item Account balance từ Binance API
    \end{itemize}
    \item Auto trading với stop loss/take profit
    \item Chu kỳ phân tích: 5 phút (có thể điều chỉnh)
\end{itemize}

\section{Lỗi Thường Gặp}

\begin{longtable}{p{5cm}|p{8cm}}
\toprule
\textbf{Lỗi} & \textbf{Giải pháp} \\
\midrule
API key invalid & Kiểm tra file cấu hình \\
\midrule
OpenAI limit & Giảm frequency hoặc check billing \\
\midrule
Balance insufficient & Nạp testnet funds \\
\bottomrule
\end{longtable}

\vspace{2cm}

\begin{center}
\textbf{⚠️ Educational Use Only - Không dùng tiền thật!}
\end{center}

\end{document}

